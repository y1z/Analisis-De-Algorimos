\documentclass{article}
\usepackage{amsmath}
\usepackage{amsfonts}
\usepackage{amssymb}

% this package is for entering code in latex 
\usepackage{listings}
\usepackage{color}

\definecolor{JustGreen}{rgb}{0,0.6,0}

% need to define this 2 first before calling maketitle in the document 
\author{Yhaliff Said Barraza}
\title{Converting Fibonacci To c++}

%settings for presenting code 
\lstset{ 
  language=C++,
  keywordstyle=\color{blue}, %color of the keywords 
  tabsize=2,
  commentstyle=\color{JustGreen},
     frame=L,
}

\begin{document}
\maketitle

\section{Fibonacci sequence}
Here is 1 of the formulas for finding a number in the \textbf{Fibonacci sequence}  

$$ X_n = X_{n-1} + X_{n-2 } \quad n > 1$$

Now lets convert this to \textbf{C++}, First will use recursion to find the number $X_n$


\subsection{Code Recursion Fibonacci}

\textbf{Note :} using this function with \textit{input value} below 35 will be relatively fast however once the \textit{input value} become bigger than 40 than you going to start to wait for some time.
\begin{lstlisting}
int RecursiveFibonacci(int Index)
{
	//! this is for handling exceptions
	if (Index == 0) { return 0; }

	if (Index == 1)
	{
		return 0;
	}
	else if (Index == 2) {
		return 1;
	}
	return RecursiveFibonacci(Index - 1) + RecursiveFibonacci(Index - 2);
}
\end{lstlisting}

\textbf{Note :} There is a faster way to get the value $X_n$ than the one above using interactions en place of recursion

% Iterative version
\subsection{Code Iterative Fibonacci }

\begin{lstlisting} 
int NonRecursiveFibonacci(int Index) {
	//! this are just the first 2 values of Fibonacci
	int FirstValue = 0;
	int SecondValue = 1;

	int Result = 0;

	if (Index == 1) { return FirstValue; }
	if (Index == 2) { return SecondValue; }

	// I'm subtracting 2 from index because it would give me 
	// a correct number, but it would be 2 ahead of the recursive function
	for (int i = 0; i < Index - 2; ++i) {

		Result = FirstValue + SecondValue;
		FirstValue = SecondValue;
		SecondValue = Result;
	}

	return Result;
}
\end{lstlisting}

\end{document}
